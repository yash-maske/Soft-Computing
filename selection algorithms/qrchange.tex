\documentclass[12pt,letterpaper]{article}
\usepackage[utf8]{inputenc}
\usepackage{geometry}
\geometry{letterpaper, left=1.5in, right=1.5in, top=1.5in, bottom=1.5in}
\usepackage{times}
\usepackage{setspace}
\usepackage{indentfirst}
\usepackage{graphicx}
\usepackage{amsmath}
\usepackage{booktabs}
\usepackage{parskip}
\usepackage{natbib}
\usepackage{url}
\usepackage{enumitem}
\usepackage[font=small,labelfont=bf]{caption}
\usepackage{tocloft}
\setlength{\parindent}{2em}
\singlespacing
\setlength{\cftbeforesecskip}{5pt}
\setlength{\cftaftertoctitle}{\vspace{5pt}}

\begin{document}

\section*{INTRODUCTION}
Urban regions of India are in fast urbanization mode, leading to growing congestion on civic infrastructure and delivery of public services. Citizens run into potholes, garbage on the street, water leaks, broken street lights and other things related to infrastructure almost on a daily basis. Although there are mechanisms for grievance redressal through local municipal bodies and authorities, the grievance redressal mechanisms suffer from ineffectiveness, lack of knowledge/awareness and a lack of citizen-oriented strategies. The existing system also suffers from a complicated complaint registration process. The complaint registration process is frequently the same for citizens - identify the department, write and sign lengthy forms, and follow the convoluted bureaucratic process. The steps required for lodging complaints dissuade citizens from participating in the process. Additionally, complaints may go unresolved because of mismanagement, illegitimate complaints, or even a lack of monitoring. This can compound frustration for citizens and increase the rift between authorities and citizens.

QR4Change aims to overcome these barriers with the implementation of a quick, transparent, and simple grievance reporting system \citep{IJSRD2025}. The project uses QR codes, which would be placed in civic hotspots, such as corners on the street, garbage collection areas, and places known for potholes. When a citizen scans the QR code, they would open a digital complaint form with the location already provided for them, thus reducing the amount of work to register their complaint. QR4Change enables users to upload images of the problem, provide a description, and categorize what the problem is with only a few clicks. To ensure that we are addressing real and reliable complaints, we validate each complaint using deep learning-based image recognition models that identify and classify defects, such as potholes and garbage dumps, for example. Similarly, a natural language processing-based urgency scoring model can help us prioritize complaints, by examining your description and measuring severity of certain concerns. This will not only prevent decision-makers from being inundated with less important or trivial matters, but it will also allow critical matters to be resolved more quickly.

From an admin side, the backend, using Express.js and MongoDB, is responsible for managing the complaints storage, tracking complaints, and forwarding the complaint to the right department. The public dashboard, built with ReactJS, also provides transparency by allowing them to view their complaints and track the status through heatmaps and filtering options. Citizens have the ability to monitor the status of their complaints, and relevant authorities can identify clusters of issues, allocate resources efficiently, and report out on performance metrics such as response time in resolving complaints.

In conclusion, QR4Change does more than simply allow for easier complaint reporting, it values trust, transparency and accountability between the citizen and authority. By providing citizen input on civic issues in an easy and convenient way while simultaneously giving authorities information to take action, the QR4Change platform furthers the mandate of smart cities and participatory governance in India.

\section*{LITERATURE SURVEY}
To effectively receive citizen complaints and provide services, the contemporary urban administrator has to identify innovative technological responses that will address the gap between the interests of the citizenry and the responsiveness of the municipality. The advance of digital grievance redressal system—the ability for citizens to indicate civic problems through a range of electronic interfaces—originates from the advent of smart city programs. This literature review summarizes the state of the research in major technological domains that underpin the QR4Change system, including text analysis for complaint processing, image classification for infrastructure improvement, smart city complaint management, and other artificial intelligence applications to civic services, as well as authentication verification.

\subsection*{Smart City Grievance Redressal Systems}
The introduction of digital complaint management systems represents a substantive paradigm shift from traditional paper-based methods to automated, transparent and effective civic engagement platforms. Recent rigorous evaluations of grievance portals have established that contemporary systems are adopting blockchain, AI and machine learning more frequently to automate and improve grievance processing systems \citep{ScienceDirect2020, InPressCo2021}. These systems underscore the important role of user experience factors, such as usability, accessibility and engagement, in assessing the effectiveness of implementation in a wider urban context.

Recent research indicates that to enhance accountability and provide real-time updates, grievance redressal systems must be integrated with larger governance platforms, such as the Centralized Public Grievance Redress and Monitoring System (CPGRAMS) \citep{ScienceDirect2020, IJARCCE2021}. Multiple studies have highlighted that implementing such systems within disadvantaged and rural contexts is challenging due to low digital literacy and technological limitations, which undermine the efficacy of these systems. However, contemporary solutions circumvent these limitations through multi-channel complaint submission systems, and user-friendly interfaces.

This digital transformation is exemplified by several large-scale platforms. In India, the Swachhata App, developed by the Ministry of Housing and Urban Affairs (MoHUA), operates across all 4,905 urban local bodies, allowing over 17 million citizens to submit geo-tagged complaints with photographic evidence. Similarly, the MCD311 app in Delhi enables citizens to report local issues, leveraging Open311 protocols to ensure interoperability and scalability. Internationally, platforms like FixMyStreet (UK) and SeeClickFix (USA) have become benchmarks for citizen-led issue reporting, demonstrating the effectiveness of publishing complaint statuses publicly to enhance transparency and accountability. The emerging use of QR codes in systems like those proposed in Chicago and by healthcare institutions in India further simplifies the initial reporting step, making it more accessible to a wider audience \citep{ASTESJ2025, ThePrint2025}. While these platforms have made significant strides, a persistent gap remains in their limited use of intelligent automation for tasks like dynamic prioritization and advanced authenticity checks—a gap that QR4Change aims to address.

\subsection*{QR Code-Based Feedback and Complaint Systems}
In recent years, the use of QR code-based feedback systems has become increasingly popular across a variety of municipal and governmental institutions. One notable implementation is the `Bharosa-QR' system used by Srinagar Police, which allows for the gathering of public comments in real time about how the police operate and provide services. Thanks to this citizen-friendly program, people will have the option of scanning QR codes at police stations and public places that connects them to a secure feedback page where they can raise their concerns and opinions about the police’s responsiveness and conduct.

The scalability of such systems is further illustrated by the QR code-based feedback mechanism introduced by Gurugram Police, which is prominently contributed to each police station in the district. The public is able to submit online feedback forms about their opinion of the police officer's conduct, service complaints, and comments such as the availability of personnel at their station and the cleanliness of their station. Another example of this scalability trend is the QR code-based smart grievance redressal system to foster citizen engagement that is implemented by Bhiwandi Nizampur City Municipal Corporation providing citizens easy methods to submit grievances against the corporation with the ability to track them in real time from the comfort of their home \citep{ThePrint2025}.

Some key benefits of these implementations are simplified complaint registration processes, real-time tracking updates, and enhanced transparency using centralized monitoring dashboards. Common features of the systems include user-friendly interfaces, one-time registration processes, systematic tracking systems, and built-in escalation protocols with accountability features.

\subsection*{Artificial Intelligence in Complaint Management}
The integration of artificial intelligence into complaint management systems has changed the way organizations respond to complaints from citizens and customers entirely \citep{LeewayHertz2025}. State-of-the-art natural language processing algorithms support contemporary AI systems, which automatically and continually monitor diverse complaint channels (e.g., chatbots, social media platforms, web form inputs and emails). These AI systems excel at extracting content and contextual analysis in real time, while NLP algorithms leverage relevant terms and phrases to classify and prioritize complaints automatically. Sentiment analysis functions, found in more sophisticated AI applications, enable systems to assess the emotional tone of consumer comments and complaints and identify and address chronic problems quickly. By examining past complaint data to find patterns and trends that can point to possible problems before they become widespread, predictive analytics further improves system capabilities. By taking a proactive stance, companies may show their dedication to ongoing development and put preventative measures into place. Sophisticated models that help complaint managers classify complaints, make choices, offer suggestions, and write suitable response letters are recent advancements in AI-assisted complaint handling. In addition to incorporating automated redress assessment capabilities that determine suitable compensation based on complaint severity, organizational standards, and individual customer circumstances, these systems guarantee fair, efficient complaint handling in compliance with regulatory requirements \citep{LexTalk2023, RapidInnovation2024, Civica2024}.

\subsection*{Research Gaps and Future Directions}
The QR4Change system attempts to fill a number of research gaps, even if the body of extant literature covers individual technology components pertinent to smart city complaint management in great detail. Instead of using these technologies as integrated solutions that take use of the synergistic impacts of combining AI, computer vision, and NLP capabilities, current systems usually implement them separately. The bulk of current research concentrates on text-based or image-based single-modal complaint processing, ignoring the increased accuracy and dependability that multi-modal systems can provide \citep{ScoreDetect2025, EcoFlow2025, IJERT2023, TheSAI2023}. The practical implementation issues of integrated smart city complaint systems are also not well studied, especially with relation to scalability, dependability, and the rates of citizen adoption in various socioeconomic circumstances. There is a great chance for systems like QR4Change to reconcile the theoretical capacities with the operational requirements of contemporary urban governance by bridging the gap between academic research and actual municipal implementation \citep{IJERT2023, TheSAI2023}. While each component of intelligent complaint management systems has been thoroughly researched, the combination of multi-modal AI processing capabilities with QR code-based submission interfaces is a new addition to the area, according to the literature review. By integrating location-aware complaint submission, multi-modal authenticity verification, intelligent complaint classification, and automated routing into a single platform created especially for municipal governance applications, the proposed QR4Change solution fills these gaps.

\section*{SYSTEM ARCHITECTURE}
\subsection*{Overview}
The QR4Change framework consists of six interrelated modules aimed at enhancing the processes of civic grievance reporting, verification, prioritization, and resolution. These modules focus on scalability, transparency, security, and usability, providing advantages to both citizens and municipal authorities. The system incorporates QR code technology, AI-powered verification, and user-friendly interfaces to tackle challenges in urban governance. Figure~\ref{fig:architecture} depicts the system architecture, with the workflow outlined in Section~\ref{sec:workflow}, complemented by user-centered UI design in Section~\ref{sec:ui_design}.

\begin{figure}[h]
    \centering
    \includegraphics[width=0.9\textwidth]{figures/system_architecture.png}
    \caption{System Architecture}
    \label{fig:architecture}
\end{figure}

\subsection*{QR Code Generation \& Placement}
Hotspot civic issue locations, including busy intersections, garbage collection sites, and dense urban populations, are assigned a unique QR code. Each QR code is pre-filled with geotagged metadata, such as latitude and longitude, and ward or zone. When a citizen scans the QR code, the citizen is automatically redirected to a form where an online complaint about the civic issue can be submitted, without the need for the citizen to enter the location. This not only makes the location more accurate, but it also saves time in the complaint submission process.

\subsection*{Citizen Reporting Interface}
The reporting interface is a responsive web application developed using ReactJS, allowing citizens to access a prefilled complaint form through QR code scanning. Users can choose an issue type (such as pothole or garbage), add an optional description, and upload a current image. The no-login feature reduces obstacles, encouraging greater civic participation.

\subsection*{Complaint Verification (CNN-based)}
Uploaded images undergo analysis through a deep learning pipeline utilizing YOLOv11n for pothole detection and InceptionV3 for garbage classification. This process guarantees that only legitimate complaints (such as potholes and garbage) are processed, thereby minimizing spam and preserving data integrity.

\subsection*{Urgency Scoring (NLP-based)}
An urgency scoring model based on NLP, utilizing a finely-tuned ELECTRA transformer, assesses complaint descriptions to categorize urgency as High (for instance, ``pothole causing accidents'') or Low (such as non-critical issues). This prioritization facilitates effective resource allocation by municipal authorities.

\subsection*{Backend Management}
The backend, developed using Express.js and MongoDB, handles the submission, routing, and tracking of complaints through RESTful APIs. Django facilitates verification and urgency assessment. MongoDB's adaptable schema accommodates user profiles, complaints, QR code metadata, and logs, providing scalability for various types of complaints.

\subsection*{User Interface Design}
\label{sec:ui_design}
The QR4Change system emphasizes a user-focused design to guarantee accessibility, ease of use, and transparency for both citizens and municipal authorities. Developed with ReactJS, the user interface is responsive, facilitating smooth interactions on both mobile and desktop platforms. The interface consists of three main components: the citizen reporting interface, the public dashboard, and the authority dashboard.

\textbf{Citizen Reporting Interface}: Accessible through QR code scanning, the citizen reporting interface offers a prefilled complaint form that includes location and department information extracted from QR code metadata. Users can choose an issue type (such as pothole or garbage) from a dropdown list, provide an optional text description, and upload a current image. The no-login feature removes registration obstacles, facilitating rapid reporting with minimal clicks. After submission, a unique Complaint ID is created, which can be shared via WhatsApp or other platforms, enabling citizens to monitor their complaint without needing authentication. The mobile-first design of the interface guarantees accessibility for users with diverse levels of digital literacy.

\textbf{Public Dashboard}: The public dashboard, which can be accessed without logging in, promotes transparency by showcasing real-time complaint data. It includes interactive heatmaps that illustrate complaint density across different regions, as well as filters for viewing data by state, district, or city. Important metrics such as resolution rates, departmental efficiency, and city rankings offer a comprehensive national perspective on the management of civic issues. Citizens have the ability to monitor individual complaints through the Complaint ID, allowing them to see timestamps and statuses (Pending, In Progress, Resolved). The user-friendly design of the dashboard fosters accountability and encourages prompt action from authorities.

\textbf{Authority Dashboard}: Local government officials utilize a secure, login-protected dashboard to oversee complaints. The platform allows for filtering based on location, urgency, type of issue, or time period, with heatmaps indicating areas with a high volume of complaints. Metrics like resolution rates and the total number of complaints per department assist in resource distribution and performance evaluation. Officials can modify the status of complaints (Pending, In Progress, Resolved), provide comments, and upload evidence of resolutions (such as photographs). The dashboard is designed to facilitate effective complaint management and support data-informed decision-making.

\textbf{Design Principles and Implementation}: The user interface follows the principles of simplicity, accessibility, and responsiveness. ReactJS facilitates dynamic rendering and real-time updates, working in conjunction with the Express.js backend and MongoDB to ensure smooth data flow. Features such as the no-login citizen interface and shareable Complaint ID boost user engagement, while the visualizations on the public dashboard foster transparency. The design of the system reduces user effort, necessitating only a few clicks for submitting and tracking complaints. Usability testing verifies intuitive navigation, guaranteeing wide accessibility for various user groups.

Figure~\ref{fig:ui} depicts essential UI elements: (a) the citizen reporting form, which includes prefilled location fields, a dropdown for issue types, and an option to upload images; (b) the public dashboard, with filters for viewing by state or district, and metrics such as resolution rates; and (c) the authority dashboard, which offers options for filtering complaints, updates on status, and regional heatmaps. These screenshots emphasize the system’s user-friendly and transparent design, promoting effective civic participation.

\begin{figure}[h]
    \centering
    \includegraphics[width=0.9\textwidth]{figures/ui_screenshots.png}
    \caption{Screenshots of QR4Change User Interface: (a) Citizen Reporting Form, (b) Public Dashboard, (c) Authority Dashboard}
    \label{fig:ui}
\end{figure}

\subsection*{Public Dashboard \& Heatmaps}
Citizens and officials will be able to view a live dashboard featuring heatmaps that indicate where complaints are concentrated. This dashboard will include various filters for the type of complaint, its location, priority level, and current status. It enhances transparency, allowing citizens to track the status of their complaints and the response rate of departments. Additionally, authorities can monitor trends related to recurring issues in specific wards, enabling them to plan community maintenance schedules more effectively.

\subsection*{System Workflow}
\label{sec:workflow}
Citizens and officials will be able to view a live dashboard featuring heatmaps that indicate where complaints are concentrated. This dashboard will include various filters for the type of complaint, its location, priority level, and current status. It enhances transparency, allowing citizens to track the status of their complaints and the response rate of departments. Additionally, authorities can monitor trends related to recurring issues in specific wards, enabling them to plan community maintenance schedules more effectively.

\begin{enumerate}
    \item \textbf{QR Code Scan (Complaint Initiation)}: Citizens identify civic issues (such as potholes and garbage accumulation) and scan a QR code placed at designated hotspots, including signboards, lamp posts, or public notice boards. Each QR code holds metadata that encompasses latitude, longitude, ward details, and department routing information, offering a direct connection to an online complaint portal.
    \item \textbf{Complaint Form Access \& Auto-Prefill}: After scanning, citizens are directed to an intuitive complaint form located within the reporting interface, which is developed using ReactJS. This form automatically populates with location and department information extracted from the QR code metadata, reducing the need for manual entry. Citizens can choose the type of issue (such as pothole or garbage), add an optional description, and upload a current image of the problem.
    \item \textbf{Complaint Verification (Deep Learning Model)}: The uploaded image is analyzed by a deep learning verification model that utilizes YOLOv11n for detecting potholes and InceptionV3 for classifying garbage. The system assesses whether the image depicts a legitimate civic issue (such as a pothole or garbage) or an invalid submission (like irrelevant content or spam). Invalid complaints are dismissed with a feedback message sent to the citizen, whereas valid complaints move on to the subsequent stage.
    \item \textbf{Urgency Scoring (NLP Model)}: The description of the text submitted by the citizen is assessed through an NLP-based urgency scoring model that employs a meticulously calibrated ELECTRA transformer. This model categorizes complaints into High urgency (for example, ``pothole causing accidents'') or Low urgency (like non-critical issues), thereby aiding in the prioritization of urgent concerns for faster resolution.
    \item \textbf{Complaint Submission \& Complaint ID Generation}: After the verification and urgency evaluation, the complaint is transmitted via an Express.js API to the backend, where it undergoes a final check for the appropriate image format, required fields, and the integrity of metadata. Upon successful submission, the system generates a unique Complaint ID, which is subsequently provided to the citizen. This ID can be shared through WhatsApp or other platforms, facilitating easy tracking without requiring user login.
    \item \textbf{Complaint Routing to Authorities}: Validated complaints, along with their urgency levels and geolocation data, are automatically routed to the relevant municipal department (e.g., Roads, Solid Waste Management) based on the nature of the issue and QR code metadata. Officials are alerted and can access complaints through a secure dashboard.
    \item \textbf{Authority Dashboard Interaction}: Municipal officials utilize a secure dashboard to oversee assigned complaints, which can be sorted by location, urgency, type of issue, or timeframe. The dashboard includes heatmaps that depict the density of complaints across various regions, along with metrics such as resolution rates and the overall number of complaints. Officials can update the status of complaints (Pending, In Progress, Resolved) and are able to add comments or evidence (like photos of resolutions) to document the actions taken.
    \item \textbf{Public Dashboard \& Transparency}: An accessible dashboard, which does not necessitate user login, displays real-time information on complaints. This features heatmaps that show the density of complaints, resolution rates broken down by state and district, metrics regarding departmental performance, and city rankings. Citizens can monitor their complaints using the Complaint ID, verify timestamps, and view status updates (including Pending, In Progress, Resolved). This degree of transparency promotes accountability and encourages swift action from officials.
    \item \textbf{Resolution \& Feedback Loop}: When an issue is tackled, the authorities update the complaint status to ``Resolved'' and may include supporting documentation, such as photographs. Citizens receive notifications via the public dashboard or the given Complaint ID and are encouraged to provide feedback to express their satisfaction or to highlight any persistent problems. This feedback system fosters continuous improvement, enhances citizen engagement, and strengthens trust in the governance framework.
\end{enumerate}

\begin{figure}[h]
    \centering
    \includegraphics[width=0.9\textwidth]{figures/workflow.png}
    \caption{Flow Chart of Proposed System}
    \label{fig:workflow}
\end{figure}

\section*{METHODOLOGY}
The proposed system utilizes computer vision, natural language processing, and web technologies to create an automatic, transparent, and scalable civic grievance redressal system. The proposed methodology comprises four components: dataset collection, verification model, urgency score model, and complaint routing \& management. These components link to each other to create an end-to-end solution.

\subsection*{Dataset Collection for Complaint Verification}
To build robust models for complaint verification a multi-modal dataset consisting of images was created by us \citep{Maske2025}.

\subsubsection*{Image Dataset (for Verification Model)}
The dataset was collected from several open-source repositories, government portals, and a field visit in Pune including regions like Kondhwa, Bibewadi, Swargate, and Market Yard. The dataset contains three classes: potholes, garbage dumps, non-relevant (spam or wrong) images. In total for pothole dataset, we had 2,966 images, which included 1,004 pothole images, and 1,962 plain road images. In terms of the garbage dataset, we collected 1,971, which included 712 garbage and 1,259 non-garbage images. Figure~\ref{fig:dataset} shows pie-chart of Dataset Distribution in percentage format. Data augmentation techniques, such as rotation, cropping, changes in brightness, and the flipping of images, were used to balance the dataset, and avoid overfitting.

\begin{figure}[h]
    \centering
    \includegraphics[width=0.9\textwidth]{figures/dataset_distribution.png}
    \caption{Dataset Distribution}
    \label{fig:dataset}
\end{figure}

\subsubsection*{Text Dataset (for Urgency Scoring Model)}
A dataset was created, comprised of citizen complaints about potholes and garbage, with an associated urgency value. Each dataset entry is composed of a complaint text and corresponding urgency label (High or Low). The determination of urgency was based on whether it included certain words (large, dangerous, or urgent for high urgency, and less intense language for low urgency). To increase diversity and coverage of real-world language, a large language model (Gemini) was also used to generate additional complaint text based on the same format, which meant more varied phrasing and wording of context.

After completion of the first phase data, the two datasets (manual + LLM-based) were combined and processed. Duplicate observations were discarded to avoid overfitting, and the data was cleaned for consistency. The resulting final dataset contained 3,785 samples of which there were 1,965 Low urgency and 1,793 High urgency samples, maintaining an equal ratio of samples in the two clusters. This final dataset was used for training and assessing the proposed classification model.

\subsection*{Verification Model}
The Verification Model ensures that only genuine complaints are processed, thereby preventing spam and irrelevant submissions.

\subsubsection*{Architecture}
A DeepLabV3-ResNet101 segmentation model \citep{Chen2018} trained on COCO was initially used to check if the uploaded image contained a road first, through layers of Convolution, ReLU activation, Pooling, Fully Connected and Softmax output. For pothole detection, the model YOLOv11n was fine-tuned \citep{Redmon2015}, whilst the detection of garbage was handled by an Inception-based classifier \citep{Szegedy2014} trained to see if severe garbage was present.

\subsubsection*{Training \& Optimization}
The custom YOLOv11n-cls model was trained on the dataset for 30 epochs using images of size 224$\times$224 with a batch size of 16. The classification component utilized Categorical Cross-Entropy as its loss function, while the YOLO detection head employed the standard YOLO detection loss for pothole localization. Training was performed using the Adam optimizer, along with learning rate tuning for optimal convergence. Model evaluation was conducted using Accuracy, Precision, Recall, and F1-score for classification and mean Average Precision (mAP) and Intersection over Union (IoU) for detection. This setup ensures both accurate classification and precise detection of potholes in the input images.

\subsubsection*{Working}
The input is the scaled and normalized image of the citizen. A multi-stage deep learning pipeline is then given this standardized image to guarantee the complaint's validity and precise classification. The first phase is semantic segmentation with DeepLabV3, which checks for the presence of a road or similar public location to confirm the image's contextual significance. This serves as a first line of defense against submissions that aren't relevant.

Depending on the initial complaint category, the image is sent to one of two specialized models after the context has been verified. The system employs YOLOv11n (You Only Look Once, version 11 nano) a cutting-edge, lightweight object detection architecture to detect potholes. YOLO divides the input image into a grid and concurrently predicts bounding boxes and class probabilities for each grid cell, as shown in Figure~\ref{fig:yolo}. Its nano variant is optimized for fast, real-time inference, enabling efficient detection of one or more potholes in a single pass through its convolutional network.

For enhanced performance, a custom YOLOv11n classification model was developed. This model accepts 224$\times$224 RGB images and extracts hierarchical features through a Backbone composed of convolutional layers and C2f blocks for multi-scale representation. To capture contextual information at multiple scales, an SPPF (Spatial Pyramid Pooling – Fast) layer was integrated, followed by a custom C2f block with 512 filters to refine high-level feature representations. The classification head performs Global Average Pooling, followed by a Linear layer and Softmax activation to output class probabilities.

\begin{figure}[h]
    \centering
    \includegraphics[width=0.9\textwidth]{figures/yolo_architecture.png}
    \caption{Architecture of the YOLO11n model for pothole detection}
    \label{fig:yolo}
\end{figure}

The system uses InceptionV3, a deep convolutional neural network known for its outstanding picture classification accuracy and processing efficiency, for garbage-related complaints. The design is based on inception modules that apply concurrent convolutional filters of different sizes (e.g., 1$\times$1, 3$\times$3, 5$\times$5), as shown in Figure~\ref{fig:inception}. The model is quite good at recognizing and categorizing various types of trash in a complicated metropolitan landscape because of its design, which enables it to collect visual information at various scales, from fine-grained textures to wider item shapes.

\begin{figure}[h]
    \centering
    \includegraphics[width=0.9\textwidth]{figures/inception_architecture.png}
    \caption{Architecture of the InceptionV3 model for garbage classification}
    \label{fig:inception}
\end{figure}

The selected model generates a likelihood score that determines the legitimacy of the complaint. Should the contribution be deemed invalid, the system will automatically reject it and inform the user of the rationale. This crucial procedure is essential to avert misuse of the system and ensure that only valid complaints are forwarded to the relevant local authorities for resolution.

\subsection*{Urgency Model}
\subsubsection*{Working}
An Urgency Scoring Model to classify citizen complaints as either High or Low urgency was developed. The dataset was comprised of 3,785 complaint texts related to pothole and garbage complaints. We preprocessed and cleaned each text and subsequently tokenized the data utilizing the ELECTRA tokenizer, which utilizes context-aware embeddings \citep{Clark2020}. Unlike more traditional approaches including TF-IDF or RNNs, ELECTRA uses transformer architecture to better capture the semantic meaning of texts related to complaints, as shown in Figure~\ref{fig:electra}.

\begin{figure}[h]
    \centering
    \includegraphics[width=0.9\textwidth]{figures/electra_architecture.png}
    \caption{Architecture of the ELECTRA model for urgency detection}
    \label{fig:electra}
\end{figure}

The model being used applies the ELECTRA-base transformer for the binary urgency classification of text. As shown in Figure~\ref{fig:electra}, it consists of a pipeline that uses preprocessing (i.e., cleaning, label encoding, duplicate removal) as the first step, followed by tokenization and embedding with a sequence-length of max 256. The lower six layers of the ELECTRA encoder are frozen to retain the pretrained features specific to linguistics, while the upper six layers will be fine-tuned for adaptation to the task. A dense layer with dropout and a softmax classifier generate the prediction for Low or High Urgency. The model is optimized using the AdamW optimizer with a learning rate of 2e-5 and weighted cross-entropy loss, and evaluated based on accuracy, precision, recall, and F1-score. The model ensures that urgent cases - such as large potholes, flooding, or garbage accumulation near hospitals - are prioritized over non-urgent cases.

\subsection*{Complaint Routing \& Management}
\subsubsection*{Routing Mechanism}
Every complaint is classified by the issue type (e.g., pothole, garbage), geolocation mapped by QR code metadata, and urgency level. The backend determines how it is assigned to a municipal authority (e.g., Roads Department, Solid Waste Department, Communications Department, or Health Department based on the issues and their supporting notes).

\subsubsection*{Authority Dashboard}
Authorities can sign in to see the complaints assigned to them. They can filter complaints by urgency, location, and type and update the status to Pending, In Progress, or Resolved, including the ability to upload proof-of-resolution photographs.

\subsubsection*{Public Dashboard (Transparency Layer)}
Citizens can have a transparent and accessible means of tracking in real-time the status of their complaints and the dashboard allow citizens to see heatmaps, metrics, and trends in complaints. The tracking of complaints promotes accountability as it ensures that authorities cannot neglect reported issues.

\subsubsection*{Feedback Loop}
When a complaint is marked as resolved, citizens get a notification and are also able to offer feedback to advise whether they find the resolution acceptable. This formal complaint lifecycle provides end-to-end accountability, reduces resolution time, and enhances citizens' engagement.

\section*{EXPERIMENTAL RESULTS}
\subsection*{Confusion Matrix Analysis}
The Confusion Matrix provides standard metrics that are used to evaluate the categorization models. The matrix evaluates performance using four key counts:
\begin{itemize}
    \item \textbf{True Positives (TP)}: Accurately anticipated positive results (e.g., real pothole found).
    \item \textbf{True Negatives (TN)}: Negative instances that were accurately anticipated (e.g., real non-pothole identified).
    \item \textbf{False Positives (FP)}: Positive situations that were mispredicted (Type I error; for example, a pothole is identified on a plain road).
    \item \textbf{False Negatives (FN)}: Type II errors that result in incorrectly anticipated negative instances, such as genuine potholes being missed.
\end{itemize}

The metrics listed below are computed using these counts:
\begin{itemize}
    \item \textbf{Accuracy}: Defined as $\frac{TP + TN}{TP + TN + FP + FN}$.
    \item \textbf{Precision (Positive Predictive Value)}: The proportion of positive identifications that were actually correct, $\frac{TP}{TP + FP}$.
    \item \textbf{Recall (Sensitivity)}: The proportion of actual positive cases that were correctly identified, $\frac{TP}{TP + FN}$.
    \item \textbf{F1-Score}: The harmonic mean of Precision and Recall, $2 \times \frac{\text{Precision} \times \text{Recall}}{\text{Precision} + \text{Recall}}$.
\end{itemize}

\textbf{YOLO11n Pothole Detection Performance}: An evaluation of the confusion matrix shown in Figure~\ref{fig:cm_pothole}, shows high classification performance with True Negatives (TN) = 147, False Positives (FP) = 7, False Negatives (FN) = 3, and True Positives (TP) = 195. This is equivalent to precision $\approx$ 0.965, recall $\approx$ 0.984, and F1-score $\approx$ 0.974, which shows low misclassification errors for both positive and negatives.

\begin{figure}[h]
    \centering
    \includegraphics[width=0.9\textwidth]{figures/cm_pothole.png}
    \caption{Confusion Matrix for pothole detection}
    \label{fig:cm_pothole}
\end{figure}

\textbf{InceptionV3 Garbage Classification Performance}: The garbage classification confusion matrix shown in Figure~\ref{fig:cm_garbage}, indicates very good performance with True Negatives (TN) = 130, False Positives (FP) = 4, False Negatives (FN) = 15, and True Positives (TP) = 127. Overall metrics give precision $\approx$ 0.96, recall $\approx$ 0.89, F1-score $\approx$ 0.92 and validation accuracy $\approx$ 0.93. Higher recall indicates there is very strong sensitivity to correctly identify actual garbage is happening, whereas lower precision indicates there is perhaps some over caution in making positive classifications.

\begin{figure}[h]
    \centering
    \includegraphics[width=0.9\textwidth]{figures/cm_garbage.png}
    \caption{Confusion Matrix for garbage detection}
    \label{fig:cm_garbage}
\end{figure}

\subsection*{ROC Curve Analysis}
The Receiver Operating Characteristic (ROC) curve illustrates the relationship between the True Positive Rate (TPR) and the False Positive Rate (FPR) across different threshold settings.
\begin{itemize}
    \item \textbf{True Positive Rate (TPR)}: This refers to the proportion of accurately predicted positive cases, which is synonymous with Recall.
    \item \textbf{False Positive Rate (FPR)}: This indicates the proportion of incorrectly predicted positive cases relative to all negative cases, $\frac{FP}{FP + TN}$.
\end{itemize}

The Receiver Operating Characteristic (ROC) curve provides a comprehensive evaluation of the classification performance of the trained YOLOv11n model. As depicted in Figure~\ref{fig:roc_pothole}, the curve consistently remains above the diagonal baseline, indicating that the model surpasses random guessing. The computed Area Under the Curve (AUC) is 0.72, reflecting a moderate degree of discriminatory power between the positive and negative classes. This suggests that the model can effectively distinguish between the two classes in approximately 72\% of cases across different decision thresholds. Although the model demonstrates significant predictive capabilities, the shape of the ROC also highlights potential trade-offs between sensitivity (true positive rate) and specificity (false positive rate). In other words, modifying the decision threshold to maximize a parametric index or using additional data to enhance classification performance would make the model even more robust.

\begin{figure}[h]
    \centering
    \includegraphics[width=0.9\textwidth]{figures/roc_pothole.png}
    \caption{ROC curve for pothole detection model}
    \label{fig:roc_pothole}
\end{figure}

Figure~\ref{fig:roc_garbage} presents the ROC curve associated with the InceptionV3 model in the context of detecting garbage. The AUC of 0.93 suggests a high level of separation between garbage and non-garbage samples. The curve indicates that there are very high sensitivity levels identified by the model, shown by the steep curve that progresses toward the upper-left corner. Overall, the model correctly identified garbage, while providing very few false positives. The model demonstrates reliable and workable automated waste detection. Given the high AUC, InceptionV3 has strong performance in both feature extraction and classification for waste detection purposes. Overall, the model shows potential for intelligent waste management.

\begin{figure}[h]
    \centering
    \includegraphics[width=0.9\textwidth]{figures/roc_garbage.png}
    \caption{ROC curve for garbage detection model}
    \label{fig:roc_garbage}
\end{figure}

\subsection*{Overall Performance}
The deep learning elements of the QR4Change system were fully tested against eight of the leading architectures to discover the best model for classifying a complaint's image. We utilized two spatial datasets (civic infrastructure) which were garbage management topic and pothole detection, representing the most prevalent urban complaints. Table~\ref{tab:performance} shows results of implemented models for garbage and pothole detection respectively using various parameters.

\textbf{Performance on the garbage dataset}: Eight models were tested on the garbage classification task. InceptionV3 achieved the best performance with a validation accuracy of 90.23\% and an F1-score of 0.90. The DenseNet121 performed similarly with 89.47\% validation accuracy and F1-score of 0.89. VGG16 achieved the highest raw accuracy of 90.98\% but lacked other metrics required for complete assessment. MobileNetV2 showed a validation accuracy of 91.73\%; however, its balanced accuracy of 49\% and F1-score of 0.48 indicated poor generalization and overfitting.

\textbf{Results for the Pothole Dataset}: The pothole classification task showcased markedly different optimal architectures, with YOLO11n yielding the highest results at 99.43\% of validation accuracy and F1-score of 0.99. InceptionV3 again performed well across both datasets with an accuracy of 96.45\% and F1-score of 0.95 for pothole. Custom CNN provided a strong baseline at 94.09\% accuracy and for traditional architectures, ResNet50 only achieved a validation accuracy of 74.00\% which indicated performance was poor overall.

\begin{table}[h]
    \centering
    \caption{Performance outcomes of implemented models}
    \label{tab:performance}
    \begin{tabular}{lcccc}
        \toprule
        Model & Dataset & Accuracy (\%) & F1-Score & Balanced Accuracy (\%) \\
        \midrule
        InceptionV3 & Garbage & 90.23 & 0.90 & - \\
        DenseNet121 & Garbage & 89.47 & 0.89 & - \\
        VGG16 & Garbage & 90.98 & - & - \\
        MobileNetV2 & Garbage & 91.73 & 0.48 & 49 \\
        YOLO11n & Pothole & 99.43 & 0.99 & - \\
        InceptionV3 & Pothole & 96.45 & 0.95 & - \\
        Custom CNN & Pothole & 94.09 & - & - \\
        ResNet50 & Pothole & 74.00 & - & - \\
        \bottomrule
    \end{tabular}
\end{table}

\subsection*{Model Comparison and Selection Rationale}
The comparative study shows clear architectural preferences for the different civic complaint categories. For the garbage category, transfer learning models pretrained on ImageNet showed the best feature extraction capabilities, as InceptionV3’s inception modules effectively captured diverse visual patterns in waste management scenarios. The model’s precision of 0.92 and recall of 0.90 indicate robust performance in positive identification and false positive minimization.

On the other hand, the pothole detection task leveraged YOLO11n's object detection architecture and was nearly flawless with a precision, recall, and F1-score of 0.99. This high-quality performance is due to YOLO's spatial attention capabilities and anchor box detection allowing YOLO to learn localized damage patterns on infrastructure from the difficult road surface images.

\subsection*{Processing Efficiency and Deployment Considerations}
The chosen models show positive computational characteristics for real-time municipal use. InceptionV3's inception modules combine the best trade-off between accuracy and inference time for relevant real-world settings and can be utilized on mobile scenarios with citizens uploading images for complaints. YOLO11n is a single-shot detection architecture that benefits rapid processing and analysis times for pothole images due to time constraints for timely infrastructure monitoring applications.

\subsection*{Cross-Dataset Generalization}
The experiments demonstrate the benefit of using a model specifically selected for the domain task of processing civic complaints. Although InceptionV3 produced consistent results across both datasets (90.23\% for garbage, 96.45\% for potholes), specialized architectures (like YOLO11n) outperformed InceptionV3 on specific tasks. This supports the multi-model approach used in QR4Change, where optimized architectures exist for each complaint category rather than one universal classifier. Future dataset expansions will include things like drainage issues, broken streetlights, and road signage issues, with unambiguous procedures for developing a model based on the current evaluation and comparison framework. The consistent comparative methods ensure the best performance across a variety of urban infrastructure complaint situations while maintaining consistent standards of evaluation for new urban civic issue types.

\subsection*{System-Level Comparative Analysis}
We carried out an experimental comparison with current municipal grievance management systems in order to fully comprehend the impact of the proposed QR4Change system. We examined a number of operational and functional factors, such as scalability, accuracy, automation, and responsiveness. A summary of the findings is shown in Table~\ref{tab:comparison}, and it is evident that the new system delivers notable improvements in automation, accuracy, and cost effectiveness in addition to offering real-time grievance resolution.

\begin{table}[h]
    \centering
    \caption{Comparative Analysis of Grievance Management Systems}
    \label{tab:comparison}
    \begin{tabular}{lcccc}
        \toprule
        System & Scalability & Accuracy & Automation & Responsiveness \\
        \midrule
        QR4Change & High & High & High & High \\
        Swachhata App & High & Medium & Medium & Medium \\
        MCD311 & Medium & Medium & Low & Medium \\
        FixMyStreet & High & Medium & Medium & High \\
        \bottomrule
    \end{tabular}
\end{table}

\subsection*{Statistical Significance and Validation}
All metrics reported are the means of multiple training runs with different random seeds for statistical reliability. The significant gaps between the highest performing models (InceptionV3, YOLO11n) and the worst performing architectures (EfficientNetB0, ResNet50) indicate an architectural preference for civic infrastructure classification problems. Models receiving an F1-score below 0.70 were removed from consideration on production usage as they provide insufficient reliability for use in municipal decision making. The validation methodology that was utilized used stratified k-fold cross-validation to evaluate performance on different types of complaints and geographies, allowing us to confidently assess real-world deployment effectiveness across different municipal settings.

\subsection*{Urgency Prediction Model Results}
The urgency prediction model was trained and validated with the ELECTRA transformer that has been fine-tuned on the complete dataset of 3,785 complaint texts (1,965 Low and 1,793 High urgency). Each complaint text was tokenized using the ELECTRA tokenizer, which generates contextual embeddings to represent the dual meaning and expressions that regard urgency in the text. To avoid overfitting for a small dataset size, the lower encoder layers of ELECTRA were frozen during training, along with class weights to balance the urgency categories. The model obtained excellent results—99.73\% Accuracy, 99.45\% Precision, 100\% Recall, and 99.72\% F1-Score. The recall score is particularly important because it demonstrates the model was confidently able to find cases with an urgent complaint. These results show that ELECTRA is successful in learning urgency that is hidden in the text descriptions and enables the reliable identification of what deserves an urgent response, e.g., huge potholes or garbage pickup around daycare.

\section*{CONCLUSION}
The QR4Change Smart City Grievance Redressal System demonstrates how AI-powered civic engagement could improve local governance. By merging QR code technology with deep learning, it improves transparency and response effectiveness. Out of eight models tested, InceptionV3 hit 90.23\% accuracy for detecting garbage and YOLO11n 99.43\% for potholes. The ELECTRA model fine-tuned achieved 99.73\% accuracy in urgency prediction and was found to be reliable for prioritization of complaints. Developed using ReactJS, Express.js, Django, and MongoDB, the system allows for real-time, scalable, and mobile-optimized operation with inference durations less than 100 milliseconds.

QR4Change improves smart city infrastructure through greater participation and service effectiveness. Location-based complaint submission eliminates geographical ambiguity, AI-based authenticity verification eliminates false reports, and automated routing optimizes resource allocation based on data. Scalability in other categories of drainage, street lighting, and road signs shows its potential for use in more extensive smart city projects. The combination of citizen-facing QR interfaces with enterprise-grade AI offers an intellectual bridge between citizens and public services, a platform for open, accountable, and responsive urban governance. QR4Change therefore enables a sustainable smart city system that normalizes multimodal AI methods for civic infrastructure monitoring and enables citizens to directly participate in processes of urban governance.

\section*{CONFLICT OF INTEREST}
We declare that there is no conflict of interest regarding the publication of this paper.

\section*{FUNDING INFORMATION}
No specific funding was received for this research. [Please update with actual funding details if applicable.]

\section*{DECLARATION OF AI TECHNOLOGY USAGE}
We declare that no Artificial Intelligence (AI) technologies or AI-assisted tools were utilized in any capacity during the writing and preparation of this article.

\bibliographystyle{plainnat}
\begin{thebibliography}{24}
\bibitem{IJSRD2025}
IJSRD (2025). Confidential Grievance Reporting System for Smart City. \url{https://ijsrd.com/Article.php?manuscript=IJSRDV12I120042}.

\bibitem{IJARCCE2021}
IJARCCE (2021). Public Complaint Sorting for Smart City using Image Processing. \url{https://ijarcce.com/wp-content/uploads/2021/06/IJARCCE.2021.105151.pdf}.

\bibitem{ScienceDirect2020}
ScienceDirect (2020). Bias in smart city governance: How socio-spatial disparities in 311. \url{https://www.sciencedirect.com/science/article/abs/pii/S2210670720307216}.

\bibitem{InPressCo2021}
InPressCo (2021). Smart Control Complaint System for Municipal Corporations. \url{http://inpressco.com/wp-content/uploads/2021/02/Paper43199-202.pdf}.

\bibitem{ASTESJ2025}
ASTESJ (2025). The Evaluation of Complaint Application that Connected to Smart City. \url{https://www.astesj.com/v05/i04/p03/}.

\bibitem{JETIR2025}
JETIR Research Journal (2025). Online Complaint Management System. \url{https://www.jetir.org/papers/JETIR2506925.pdf}.

\bibitem{IJRPR2025}
International Journal of Research Publication and Reviews (2025). Smart City Grievance Redressal System Research. \url{https://ijrpr.com/uploads/V6ISSUE4/IJRPR41688.pdf}.

\bibitem{ThePrint2025}
The Print (2025). Civic body in Maharashtra introduces QR code-based complaint redressal system. \url{https://theprint.in/india/civic-body-in-maharashtra-introduces-qr-code-based-complaint-redressal-system/2575388/}.

\bibitem{LeewayHertz2025}
LeewayHertz (2025). AI in customer complaint management: Use cases, benefits. \url{https://www.leewayhertz.com/ai-in-complaint-management/}.

\bibitem{LexTalk2023}
LexTalk World (2023). How Artificial Intelligence can help in developing Complaint Management Software. \url{https://www.lextalk.world/post/how-artificial-intelligence-can-help-in-terms-of-developing-complaint-management-software}.

\bibitem{RapidInnovation2024}
Rapid Innovation (2024). AI Complaint Management Guide 2025. \url{https://www.rapidinnovation.io/post/ai-agents-for-complaint-management-benefits-use-cases-and-challenges}.

\bibitem{Civica2024}
Civica (2024). Enhancing Complaints Management with Artificial Intelligence. \url{https://www.civica.com/en-gb/insights/enhancing-complaints-management-with-artificial-intelligence/}.

\bibitem{ScoreDetect2025}
ScoreDetect (2025). UGC Authenticity: 7 Verification Methods. \url{https://www.scoredetect.com/blog/posts/ugc-authenticity-7-verification-methods}.

\bibitem{Scoop2025}
Scoop.it (2025). 7 Steps to Determine Authentic User-Generated Content (UGC). \url{https://blog.scoop.it/2025/03/06/7-steps-to-determine-authentic-user-generated-content-ugc/}.

\bibitem{EcoFlow2025}
EcoFlow (2025). Authenticity and User-Generated Content (UGC). \url{https://www.ecoflow.com/us/blog/authenticity-user-generated-content-guide}.

\bibitem{IJERT2023}
IJERT (2023). Implementing NLP to Categorize Grievances Received Via A Voice Input Mechanism. \url{https://www.ijert.org/implementing-nlp-to-categorize-grievances-received-via-a-voice-input-mechanism}.

\bibitem{TheSAI2023}
The SAI (2023). Utilizing NLP to Optimize Municipal Services Delivery Using a Novel Framework. \url{https://thesai.org/Downloads/Volume16No2/Paper_78-Utilizing_NLP_to_Optimize_Municipal_Services_Delivery.pdf}.

\bibitem{Maske2025}
Maske, Y., Jakate, S., Thakare, C., Lokhande, S. (2025). Urban Civic Issues Image Dataset: Potholes and Garbage (QR4Change). Mendeley Data, V2, doi: 10.17632/zndzygc3p3.2.

\bibitem{Chen2018}
Chen, L.-C., Zhu, Y., Papandreou, G., Schroff, F., Adam, H. (2018). Encoder-Decoder with Atrous Separable Convolution for Semantic Image Segmentation. \url{https://arxiv.org/abs/1802.02611}.

\bibitem{Redmon2015}
Redmon, J., Divvala, S., Girshick, R., Farhadi, A. (2015). You Only Look Once: Unified, Real-Time Object Detection. \url{https://arxiv.org/abs/1506.02640}.

\bibitem{Szegedy2014}
Szegedy, C. et al. (2014). Going Deeper with Convolutions. \url{https://arxiv.org/abs/1409.4842}.

\bibitem{Clark2020}
Clark, K., Luong, M.-T., Le, Q. V., Manning, C. D. (2020). ELECTRA: Pre-training Text Encoders as Discriminators Rather Than Generators. \url{https://arxiv.org/abs/2003.10555}.

\bibitem{IJARSCTa}
IJARSCT (n.d.). Sorting of Civic Issues using Machine Learning with Image Processing. \url{https://ijarsct.co.in/Paper1047.pdf}.

\bibitem{IJARSCTb}
IJARSCT (n.d.). Public Issues Sorting Using Image Processing with Machine Learning. \url{https://ijarsct.co.in/Paper1270.pdf}.
\end{thebibliography}

\end{document}